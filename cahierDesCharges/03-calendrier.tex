\chapter*{Calendrier et Délivrables}
\addcontentsline{toc}{chapter}{Calendrier et Délivrables}
\markboth{Calendrier Délivrables}{Calendrier et Délivrables}
\label{chap:Calendrier et Délivrables}

\begin{itemize}
    \item  \textbf{ven. 28/10/16 18h - Choix du sujet} 
    \begin{itemize}
        \item Document PDF sur le Campus Virtuel comprenant :
        \begin{itemize}
            \item La composition du groupe
            \item Une description du cahier des charges du projet (descriptif client)
            \item L’URL du repository Github avec page Wiki jour
        \end{itemize}
    \end{itemize}
    \item \textbf{ven. 11/11/16 18h Diagramme de classe UML}
    \begin{itemize}
        \item Le diagramme UML du modéle de l’application au format PDF, sur le Campus Virtuel ET sur le Wiki Github
    \end{itemize}
    \item \textbf{ven. 18/11/16 18h - Implémentation du modèle}
    \begin{itemize}
        \item Chaque étudiant du groupe soumet une classe complète du package modele, dûment spécifiée et testée, sur le Campus Virtuel. Le repository Github doit être jour avec le code correspondant.
    \end{itemize}
    \item \textbf{lun. 28/11/16 Séance TP - Démo de la vue}
    \begin{itemize}
        \item Les étudiants font une démo des interactions possibles avec le modèle depuis une interface console (ligne de commande).
    \end{itemize}
     \item \textbf{mer. 23/12/16 12h - Remise du projet}
     \begin{itemize}
         \item Page Wiki du Github jour avec :
         \begin{itemize}
             \item Composition du groupe
             \item Cahier des charges/descriptif
             \item Version finale du diagramme UML du modèle
             \item Mode d’emploi pour installer et utiliser l’application
             \item Pointeur vers les délivrables intermédiaires et finaux
         \end{itemize}
         \clearpage
         \item Sur le Campus Virtuel + copie papier remettre au professeur lors de la démo, un rapport comprenant :
         \begin{itemize}
             \item Le cahier des charges
             \item Le diagramme UML et son explication éventuelle
             \item Les choix d’implémentation effectués
             \item Les difficultés rencontrées
             \item Les pistes d’améliorations éventuelles
             \item Une conclusion individuelle de chaque membre du groupe, détaillant ses apports et son vécu personnel lors de la réalisation du projet
         \end{itemize}
     \end{itemize}
     \item \textbf{Défense finale}
     \begin{itemize}
         \item La défense du projet aura lieu durant la session de janvier. Elle consiste en une démo de l’application sur machine (pas de projection prévue).
         \item Les étudiants apportent une version imprimée du rapport cette occasion (noir et blanc, agrafé, pas de reliure, de papier glacé ou de couverture plastique).
         \item Le code source et le rapport doivent être identiques ceux remis lors de l’échéance de fin de semestre (Campus Virtuel et commits Github faisant foi).
     \end{itemize}
\end{itemize}


%%% Local Variables: 
%%% mode: latex
%%% TeX-master: "cahierDesCharges"
%%% End: 
