\chapter{Déroulement d'une partie}
\label{sec:deroulementdUnePartie}

\section{Le menu principal}

La première interaction entre l'utilisateur consistera à choisir entre :

\begin{itemize}
    \item Jouer
    \item Tutoriel
    \item Réglages
    \item Informations / Crédits
\end{itemize}

\subsection{Jouer}
L'utilisateur aura le choix entre :
\begin{itemize}
    \item Nouvelle Partie
    \item Charger une partie
\end{itemize}
\subsubsection{Nouvelle Partie}
\begin{itemize}
    \item Sélection du type de réseau social\footnote{Afin de permettre un développement du jeu par la suite, nous prévoyons déjà de la place pour d'autres mode de jeux}
\end{itemize}

\subsection{Tutoriel}
L'utilisateur aura le choix entre :
\begin{itemize}
    \item Manuel du jeu
    \item \sout{Dictaticiel} (Une fois que le jeu sera totalement développé)
\end{itemize}
\clearpage
\subsection{Réglages}
Les choix seront les suivants :
\begin{itemize}
    \item Son (GUI)
    \item Musique (GUI)
    \item Pause automatique
    \item Langage
\end{itemize}

\section{Le jeu}

\subsection{Sélection de la difficulté}
\begin{itemize}
    \item Régulier
    \item Normal
    \item Brutal
\end{itemize}
\subsection{Choix du nom du réseau social}
\subsection{Le début de l'histoire}
Lorsque la partie commence, il vous faut choisir un marché de départ. L'Europe et l'Amérique du nord étant les marchés les plus accessibles à votre réseau social, il en sera plus facile de commencer dans l'un de ces deux territoires. Chaque marché a ses propres avantages et inconvénients !

Dès que celui-ci a été choisi, la partie commence ! C'est votre soirée de lancement et vous obtiendrez vos premiers utilisateurs, ainsi qu'une base d'argent pour commencer à vous développer.

\subsection{La carte du marché mondial}
C'est l'écran principal du jeu. Pour la version graphique, il s'agit d'une carte du monde où les pays sont regroupés pour former un marché\footnote{Ex : Europe, Amérique du Nord, Asie, ...}. Graphiquement, il y aura une animation entre les différents marchés sous forme de flux de données représentant internet.



\subsection{Les actions}
\subsubsection{Menu Gestion}
A travers ce menu, vous aurez accès aux différentes améliorations :
\begin{itemize}
            \item \textbf{Growth} : Qui permet d'augmenter le nombre d'utilisateur par le biais d'actions marketing
            \item \textbf{Security} : Qui permet d'améliorer la sécurité de votre réseau social afin de ne pas subir des attaques d'autres états ou concurrents
            \item \textbf{Black Ops} : Cette catégorie d'actions consiste à faire du \textbf{Lobbying} auprès d'états pour assouplir la régulation du marché, de signer des contrats secrets avec des agences de renseignement,...
        \end{itemize}

\subsubsection{Menu Monde}
Grâce à ce menu, vous aurez accès à toutes les informations utiles concernant le monde et votre progression.

\subsubsection{Évènements Marketings}
Les évènements marketing apparaissent à l'écran de manière régulière, il suffit d'effectuer l'action demandée (GUI \& console) dessus pour l'obtenir. Il augmentera le nombre d'utilisateurs et vous apportera de l'argent.

\subsubsection{Boost}

Il vous sera possible d'ajouter des modes de transmissions à votre réseau pour étendre et gagner du terrain. Vous recevrez au fil de votre partie du crédit pour acheter ces dites transmissions (ex : bouche à oreille, pub, placements de produits...)

Il y a également différents types de fonctionnalités qui auront des niveaux que vous pourrez acheter grâce à votre crédit.

\subsubsection{Mali}

\subsubsection{Réglages}
\begin{itemize}
            \item Paramètres
            \item Sauver \& quitter
\end{itemize}
\subsubsection{Live Feed}
Le liveFeed contient des informations sur le monde, celles-ci peuvent vous donner des indices sur une faille à exploiter ainsi que sur l'avancement de vos concurrents.
\subsection{Utilisateurs Actif \& Argent}
Affiche le nombre d'utilisateurs actifs de votre réseau ainsi que votre compte en banque.
\subsection{Date}
Affiche la date actuelle, vous pouvez également augmenter la vitesse du jeu.

\section{Le jeu en CLI}
Le principe, de toute évidence, reste le même.

En interface console, la présentation du monde se fera par barre de remplissage qui représenteront les différents marchés \footnote{Source : wikipédia}: 
    \begin{itemize}
                \item zone Amérique du Nord : 579 millions
                \item zone Amérique du Sud: 422,5 millions
                \item zone Europe : 743,1 millions 
                \item zone Asie-Ouest + Sud + Centrale : 18.089.853.000
                \item zone Asie-Est + Sud-Est : 2.230.503.000
                \item zone Corée du nord : 24,9 millions 
                \item zone Russie: 143,5 millions 
    \end{itemize}


Ces barres de remplissage seront rafraîchies régulièrement au fur et à mesure de l'infection de votre réseau social dans le monde. Une visualisation de ces dites barres sera lancée et actualisée. Il vous sera toujours possible de ré-afficher ces barres en 

Concernant les évènements du jeu, un message apparaîtra sur la console. Il faudra alors presser la touche demandée dans le temps impartis.

Pour effectuer les différentes opérations possibles, vous allez devoir entrer des lignes de commande prévues à cet effet. Une page help sera à disposition de l'utilisateur en cas de doute. 

\begin{figure}
\begin{center}
\includegraphics{images/htop.png}
\end{center}
\caption{No caption for now }
\label{example}
\end{figure}


%%% Local Variables: 
%%% mode: latex
%%% TeX-master: "cahierDesCharges"
%%% End: 
