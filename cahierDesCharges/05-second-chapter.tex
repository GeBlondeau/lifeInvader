\chapter{Les exigences fonctionnelles}
\label{sec:exigencesFonctionnelles}

\section{Le menu princpal}

La première interaction entre l'utilisateur consistera à choissir entre :

\begin{itemize}
    \item Jouer
    \item Tutoriel
    \item Réglages
    \item Informations / Crédits
\end{itemize}

\subsection{Jouer}
L'utilisateur aura le choix entre :
\begin{itemize}
    \item Nouvelle Partie
    \item Charger une partie
\end{itemize}
\subsubsection{Nouvelle Partie}
\begin{itemize}
    \item Sélection du type de réseau social\footnote{Afin de permettre un développement du jeu par la suite, nous prévoyons déjà de la place pour d'autres mode de jeux}
\end{itemize}

\subsection{Tutoriel}
L'utilisateur aura le choix entre :
\begin{itemize}
    \item Manuel du jeu
    \item \sout{Dictaticiel} (Une fois que le jeu sera totalement développé)
\end{itemize}
\clearpage
\subsection{Réglages}
Les choix seront les suivants :
\begin{itemize}
    \item Son (GUI)
    \item Musique (GUI)
    \item Pause automatique
    \item Langage
\end{itemize}

\section{Le jeu}

\subsection{Sélection de la difficulté}
\begin{itemize}
    \item Régulier
    \item Normal
    \item Brutal
\end{itemize}
\subsection{Choix du nom du réseau social}
\subsection{Le début de l'histoire}
Lorsque la partie commence, il vous faut choissir un pays de départ. L'europe et l'amérique du nords sont les marchés les ouvert à votre réseau social. Il sera donc plus facile de commencer dans l'un de ces deux territoires. Chaques marché à ses propres avantages et inconvéniant !

\subsection{La carte du marché mondial}
C'est l'écran principal du jeu, pour la version graphique il s'agit d'une carte du monde où les pays sont regroupé pour former un marché\footnote{Ex : Europe, Amérique du Nord, Asie, ...}. Graphiquement, il y aura une animations entre les différents marchés sous forme de flux de donnée représentant internet.




%%% Local Variables: 
%%% mode: latex
%%% TeX-master: "cahierDesCharges"
%%% End: 