\chapter{Les exigences fonctionnelles}
\label{sec:exigencesFonctionnelles}

\section{Le menu princpal}

La première interaction entre l'utilisateur consistera à choissir entre :

\begin{itemize}
    \item Jouer
    \item Tutoriel
    \item Réglages
    \item Informations / Crédits
\end{itemize}

\subsection{Jouer}
L'utilisateur aura le choix entre :
\begin{itemize}
    \item Nouvelle Partie
    \item Charger une partie
\end{itemize}
\subsubsection{Nouvelle Partie}
\begin{itemize}
    \item Sélection du type de réseau social\footnote{Afin de permettre un développement du jeu par la suite, nous prévoyons déjà de la place pour d'autres mode de jeux}
\end{itemize}

\subsection{Tutoriel}
L'utilisateur aura le choix entre :
\begin{itemize}
    \item Manuel du jeu
    \item \sout{Dictaticiel} (Une fois que le jeu sera totalement développé)
\end{itemize}
\clearpage
\subsection{Réglages}
Les choix seront les suivants :
\begin{itemize}
    \item Son (GUI)
    \item Musique (GUI)
    \item Pause automatique
    \item Langage
\end{itemize}

\section{Le jeu}

\subsection{Sélection de la difficulté}
\begin{itemize}
    \item Régulier
    \item Normal
    \item Brutal
\end{itemize}
\subsection{Choix du nom du réseau social}
\subsection{Le début de l'histoire}
Lorsque la partie commence, il vous faut choisir un marché de départ. L'Europe et l'Amérique du nord étant les marchés les plus accessibles à votre réseau social, il en sera plus facile de commencer dans l'un de ces deux territoires. Chaque marché a ses propres avantages et inconvénients !

Dès que celui-ci a été choisi, la partie commence ! C'est votre soirée de lancement et vous obtiendrez vos premiers utilisateurs, ainsi qu'une base d'argent pour commencer à vous développer.

\subsubsection{Évènements Marketings}
Les évènements marketing apparaissent à l'écran de manière régulière, il suffit de cliquer\footnote{En CLI, un message apparaîtra sur la console, il faudra presser la touche demandée dans le temps imparti} dessus pour l'obtenir. Il augmentera le nombre d'utilisateurs et vous apportera de l'argent.

\subsection{La carte du marché mondial}
C'est l'écran principal du jeu. Pour la version graphique, il s'agit d'une carte du monde où les pays sont regroupés pour former un marché\footnote{Ex : Europe, Amérique du Nord, Asie, ...}. Graphiquement, il y aura une animation entre les différents marchés sous forme de flux de données représentant internet.



\subsection{Les actions possibles}
\subsubsection{Menu Gestion}
A travers ce menu, vous aurez accès aux différentes améliorations :
\begin{itemize}
            \item \textbf{Growth} : Qui permet d'augmenter le nombre d'utilisateur par le biais d'actions marketing
            \item \textbf{Security} : Qui permet d'améliorer la sécurité de votre réseau social afin de ne pas subir des attaques d'autres états ou concurrents
            \item \textbf{Black Ops} : Cette catégorie d'actions consiste à faire du \textbf{Lobbying} auprès d'états pour assouplir la régulation du marché, de signer des contrats secrets avec des agences de renseignement,...
        \end{itemize}

\subsubsection{Menu Monde}
Grâce à ce menu, vous aurez accès à toutes les informations utiles concernant le monde et votre progression.

\subsubsection{Réglages}
\begin{itemize}
            \item Paramètres
            \item Sauver \& quitter
\end{itemize}
\subsubsection{Live Feed}
Le liveFeed contient des informations sur le monde, celles-ci peuvent vous donner des indices sur une faille à exploiter ainsi que sur l'avancement de vos concurrents.
\subsection{Utilisateurs Actif \& Argent}
Affiche le nombre d'utilisateurs actifs de votre réseau ainsi que votre compte en banque.
\subsection{Date}
Affiche la date actuelle, vous pouvez également augmenter la vitesse du jeu.


%%% Local Variables: 
%%% mode: latex
%%% TeX-master: "cahierDesCharges"
%%% End: 