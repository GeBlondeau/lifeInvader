\chapter{Préconisations Générales}
\label{sec:preconisationsGenerales}

\section{Développement}
Les différents codes du jeu seront organisés suivant l'architecture MVC afin de garantir une facilité de lecture et un meilleur travail en équipe.

Pour le moment, nous n'avons pas encore défini de besoins spécifiques concernant des API.

Concernant l'interface graphique, sauf si nous trouvons une meilleure solution, nous allons utiliser la librairie \href{https://libgdx.badlogicgames.com/}{LibGDX\footnote{https://libgdx.badlogicgames.com/}}

\subsection{Langage}
L'application sera en Français et Anglais. Le code sera organisé de façon à respecter les techniques de localisation afin de pouvoir facilement ajouter d'autres langues.

\subsection{Client/Serveur}
Pour un bon fonctionnement, le jeu est divisé en 2 parties principales:
\begin{itemize}
    \item Le client
    
    Le client s'occupe de tout ce qui est "jouable" et transmet les commandes au serveur qui lui, organisera le jeu.
    \item Le Serveur 
    
    Le serveur s'occupe de gérer le monde, les évènements qui s'y produisent et renvois aux clients connectés les informations sur la partie en cours.
\end{itemize}



%%% Local Variables: 
%%% mode: latex
%%% TeX-master: "cahierDesCharges"
%%% End: 