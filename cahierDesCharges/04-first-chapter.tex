\chapter{Présentation générale du projet}
\label{chap:premierchapitre}

\section{Le pitch}
\textbf{Life Invader} est un jeu dans lequel vous incarnerez un nouveau réseau social. Celui à pour but de prendre le controle sur toute les informations privée de chaques personnes sur terre afin de servir un but obscur !

\section{Description de l'application}

Vous venez de créer votre nouveau réseau social. Il est temps de le développer, d'augmenter le nombre d'utilisateurs actif. Pour cela, tout les coups sont permis ! Négociez des contrats publicitaires, faite un pacte avec une nation pour obtenir des fonds contre un accès aux données de vos membres. Créez une association pour fournir internet dans des pays du tiers-monde afin que leurs habitant puisse avoir accès à vos produits !

Plusieurs type de victoire sont possible ! Arriverez vous à promouvoir votre réseau à l'ensemble de la population mondiale active ? Deviendrez-vous la société avec le plus grand capital au monde ? Découvrez les autres types de victoire en harpentant Life Invader !

\section{Cible}

Tout les détenteurs d'un ordinateur équipé de JAVA. La cible est joueur occasionnel, qui désire jouer rapidement ou prendre son temps pour aller plus loin, elle représente les jeunes de 15 à 45 ans

\section{Fonctionnalités principales}

\begin{itemize}
    \item L'interface Graphique (GUI) sera fortement inspiré de celui du jeu PlagueInc. (Figure \ref{plagueIncMap})
    \item Concernant l'interface console, celle-ci sera composé de différents menus afin d'obtenir des informations sur le monde et des repercussions des actions sur celui-ci
    
    \item Il faudra gérer votre réputation pour que les utilisateurs continuent à s'inscrire et qu'ils restent actifs
    
    \item A travers les différents menus
    
    \begin{itemize}
        \item il est possible de gérer son réseau social à travers trois type d'actions
    
        \begin{itemize}
            \item \textbf{Grwoth} : Qui permet d'augmenter le nombre d'utilisateur par le bias d'actions marketting
            \item \textbf{Security} : Qui permet d'améliorer la sécurité de votre réseau social afin de ne pas subir des attaques d'autres états ou conccurent
            \item \textbf{Black Ops} : Cette catégorie d'actions consiste à faire du \textbf{Lobbying} auprès d'états pour assouplir la régulation du marché, de signer des contrats secrets avec des agences de renseignement,...
        \end{itemize}
        \end{itemize}
        \item{Un \textbf{live feed} interactif pour augmenter le réalisme du jeu. Celui-ci évolue en fonctions des actions de l'utilisateur}
        \item Pour faire évoluer le réseau, il faut dépenser de l'argent. Celui-ci est gagné de différentes manière : 
        \begin{itemize}
            \item Suivant les mise à jours de votre réseau, vous obtiendrez régulièrement des revenus de manière automatique
            \item Périodicement, des actions marketting aurions lieu partout dans le monde, cliquez dessus et vous obtiendrez un bonus d'utilisateur et d'argent
            \item Grâce au menu blackops vous obtiendrez certaines facilités (Moins de taxes, ...) et également une grosse compensation d'argent. Mais votre réputation en prendra un coup
        \item Tout les réseaux sociaux présent dans la partie, rencontrerons périodiquement des attaques de la part d'états, de hackers et concurrents.
        \item Un concurrent commencera en même temps que vous (AI ou autre joueur), il disposera exactement des même fonctionalité, avantage/défaut que les vôtres
    \end{itemize}
    
\end{itemize}
\begin{figure}
\begin{center}
\includegraphics{images/plagueIncMap.jpg}
\end{center}
\caption{L'interface du jeu Plague Inc - \copyright NDemic Creations }
\label{plagueIncMap}
\end{figure}
%%% Local Variables: 
%%% mode: latex
%%% TeX-master: "cahierDesCharges"
%%% End: 