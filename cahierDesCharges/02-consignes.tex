\chapter*{Consignes \& liens utiles}
\addcontentsline{toc}{chapter}{Consignes \& liens utiles}
\markboth{Consignes}{Consignes}
\label{chap:Consignes}
%\minitoc

\begin{enumerate}[I]
    \item Le projet doit répondre aux contraintes suivantes :
    \begin{itemize}
        \item Une \textbf{application utilitaire} ou un \textbf{jeux}
        \item Doit respecter l'architecture \textbf{MVC} avec \textbf{deux interfaces}
        \begin{itemize}
            \item Console (\textbf{CLI})
            \item Interface graphique (\textbf{GUI})
        \end{itemize}
        \item Doit comporter une communication \textbf{Soket} ou une interaction \textbf{JDBC}
    \end{itemize}   
    \item Gestion de l'équipe/projet
    \begin{itemize}
        \item Le code doit être  \textbf{versionné sur Github} avec des commits fréquents
        \item Chaque étudiant doit participer de manière \textbf{équitable} au projet
        \item Les étudiants doivent suivre une démarche \textbf{TDD}
        \item Le \textbf{code} doit être \textbf{de qualité}
        \item La partie modèle de l'application doit être couverte par des \textbf{tests unitaires}
        \item Une \textbf{convention de codage} est définie et respectée
        \item Les \textbf{échéances} intermédiaires doivent être \textbf{respectées}
        \item Le Wiki Github a été correctement rempli et mis jour tout au long du projet
        \end{itemize}
\end{enumerate}

\textbf{Liens utiles}
\begin{itemize}
    \item \textbf{Repository Github} : \url{https://github.com/YoungChrisV/Life-Invader}
    \item \textbf{Wiki Github} : \url{https://github.com/YoungChrisV/Life-Invader/wiki}
    \item \textbf{Github Page} : \url{lifeinvader.chrisv.be}
    \item \textbf{Issue tracker} : \url{https://github.com/YoungChrisV/Life-Invader/issues}
\end{itemize}

\tableofcontents

\clearpage


%%% Local Variables: 
%%% mode: latex
%%% TeX-master: "cahierDesCharges"
%%% End: 
